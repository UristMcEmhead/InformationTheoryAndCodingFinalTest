\documentclass[../main.tex]{subfiles}

\begin{document}
\setlength{\parskip}{0em}

$f(x_1, .., x_n) = \overline{\overline{x_1} \oplus x_2} \oplus x_3 \oplus .. x_{n-2} \oplus \overline{\overline{x}_{n-1} \oplus x_{n}}$

Таким образом, необходимо найти $d_{min}$ - расстояние Хэмминга. 

$\overline{\overline{x_1} \oplus x_2} \oplus x_3 \oplus .. x_{n-2} \oplus \overline{\overline{x}_{n-1} \oplus x_{n}} = x_1\oplus x_2 \oplus .. \oplus x_n$ (Вспоминаем дискретку, и что $\overline{x} \oplus 1 = x$)

Данная функция принимает значение 1, если кол-во единиц в нечётное.

Рассмотрим $d_{min}=1$. В таком случае, у нас существует два кода, которые различаются ровно в одной позиции. Но тогда, в оном из них четное кол-во единиц, а в другом нечетное (сдвигами получить метрику 1 мы не можем), что противоречит условие характеристической функции $=> d_{min} \neq 1$

Рассмотрим $d_{min}=2$. Для док-ва возможности равенства метрике двум, нам достаточно привести пример двух кодов, разлиающихся в двух позициях. в нашем случае это $10..00; 10..11 => \exists X, что d_{min} =2$

Тогда определим сколько ошибок обнаруживает и исправляет код.

Код обнаруживает $d_{min} - 1 = 1$ ошибку, и исправляет $\frac{d_{min} -1}{2} =0$ (т.к. $ k \in N)$

\end{document}