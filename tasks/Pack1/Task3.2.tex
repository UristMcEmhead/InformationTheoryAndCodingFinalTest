\documentclass[../main.tex]{subfiles}
\setlength{\parskip}{0em}

\begin{document}

Пропускная способность равна $\sup{\{I(X;Y)\}} = \sup\{H(Y)-H(Y|X)\} = \sup\{H(Y)\} - H(1-\alpha, \alpha)$, так как величина H(Y|X) зависит только от характеристик канала)

Найдем $\sup\{H(Y)\}$. $H(Y)$ согласно теореме о максимальном значении энтропии, $H(Y)$ будет максимально, при равных вероятностях. Иначе говоря, $\overline{p_{0}}(y)=(0.5,0.5)$. 

Проверим существование такого канала. Для этого решим систему уравнений, состоящую из уравнений полных вероятностей для $p(x); p(y_i) =0.5; i=(\overline{1,2})$ (элементарная вещь - приводить не буду). 

Решив систему получим, что $p(x_1)=p(x_2)$, и так как $\sum\limits_{x \in X}p(x)=1$, тогда $p(x_1)=p(x_2)=0.5 \Rightarrow$ ситуация когда $y_1=0.5; y_2=0.5$ возможно для ДСК с заданными параметрами $\Rightarrow$ пропускная способность ДСК определяется формулой $1-H(\alpha, 1-\alpha)$

\end{document}