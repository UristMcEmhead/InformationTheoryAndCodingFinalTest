\documentclass[../main.tex]{subfiles}
\setlength{\parskip}{0em}

\begin{document}

Насколько я понял функция уже в виде полинома Жегалкина, и упростить её не получится. Так что F тому, кому попалось задание

% $f(x_1,x_2,..,x_{2n}) = x_1x_2x_3..x_{2n} \oplus x_1\overline{x}_2x_3\overline{x}_4..\overline{x}_{2n} \oplus \overline{x}_1\overline{x}_2\overline{x}_3..\overline{x}_{2n}$ \newline
% Т.к. операция $\oplus$ коммутативна, рассмотрим $x_1x_2x_3..x_{2n} \oplus \overline{x}_1\overline{x}_2\overline{x}_3..\overline{x}_{2n}$
% $$x_1x_2x_3..x_{2n} \oplus \overline{x}_1\overline{x}_2\overline{x}_3..\overline{x}_{2n} = 
% (x_1x_2x_3..x_{2n} \vee \overline{x}_1\overline{x}_2\overline{x}_3..\overline{x}_{2n}) \wedge (\overline{x_1x_2x_3..x_{2n}} \vee (\overline{\overline{x}_1\overline{x}_2\overline{x}_3..\overline{x}_{2n}})$$

% 1)$(\overline{x_1x_2x_3..x_{2n}} \vee (\overline{\overline{x}_1\overline{x}_2\overline{x}_3..\overline{x}_{2n}}) 
% = \overline{x}_{1} \vee \overline{x}_{2} \vee \overline{x}_{3} \vee ..\vee \overline{x}_{2n} \vee x_1 \vee x_2 \vee x_3 \vee .. \vee x_{2n} = 1$

% 2)$$

\end{document}