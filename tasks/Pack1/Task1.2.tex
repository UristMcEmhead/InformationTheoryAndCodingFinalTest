\documentclass[../main.tex]{subfiles}
\setlength{\parskip}{0em}

\begin{document}

\noindent
$k = 3;\ n = 6;\ GF(2)$ 

\noindent
$
x_1=u_1\newline
x_2=u_2\newline
x_3=u_3\newline
x_4=x_1\oplus x_3\newline
x_5=x_2\oplus x_1\newline
x_6=x_1\oplus x_2\oplus x_3
$

Построим порождающую матрицу. 

Это расписывать необязательно - написано для понимания:
\begin{enumerate}
    \item У нас $n-k = 3$, то есть мы получаем 3 линейно независимых строки.
    \item $x_1=u_1$, тогда первый столбец матрицы имеет вид $100$
    \item $x_2=u_2$, тогда второй столбец матрицы имеет вид $010$
    \item $x_3=u_3$, тогда третий столбец матрицы имеет вид $001$
    \item $x_4=x_1\oplus x_3$,\ тогда четвертый столбец матрицы имеет вид $101$
    \item $x_5=x_2\oplus x_1$,\ тогда пятый столбец матрицы имеет вид $110$ 
    \item $x_6=x_1\oplus x_2\oplus x_3$,\ тогда четвертый столбец матрицы имеет вид $111$
\end{enumerate}

\begin{equation*}
    G = \left( 
    \begin{array}{cccccc}
         100111  \\
         010011  \\
         001101
    \end{array}
    \right)
\end{equation*} 

По виду матрицы - видим, что код систематический. 
\begin{equation*}
    G_1 = \left( 
    \begin{array}{ccc}
         111  \\
         011  \\
         101
    \end{array}
    \right)
\end{equation*} 

Построим проверочную матрицу. \newline
$H=[-G^{T}_{1},I_{n-k}]$, где через $G^{T}_{1}$ -обозначена транспонированная матрица для $G_1$

\begin{equation*}
    H = \left( 
    \begin{array}{cccccc}
         101100  \\
         110010  \\
         111001
    \end{array}
    \right)
\end{equation*} 

Тогда проверочная матрица принимает вид

\begin{equation*}
    H^T = \left( 
    \begin{array}{ccc}
         111 \\
         011 \\
         101 \\
         100 \\
         010 \\
         001
    \end{array}
    \right)
\end{equation*} 

Посторим таблицу синдромов

\begin{center}
    \begin{tabular}{|c|c|}
    \hline
     001 & 000001\\ 
     010 & 000010\\ 
     011 & 010000\\ 
     100 & 000100\\ 
     101 & 001000\\ 
     110 & 000110\\ 
     111 & 100000\\\hline
\end{tabular}
\end{center}

Проверим работу алгоритма. \newline $\overline{y}=110111$ \newline
$\overline{y}H^T=011\ \overline{x}=110111 \oplus 010000 = 100111 $

Из таблицы видно, что код исправляет 6 одиночных ошибок, и 1 двойную ошибку. Вычислим вероятность ошибки декодирования для ДСК:

$p_\varepsilon=1-(1-p)^6 - 6p(1-p)^5 - p^2(1-p)^4$


\end{document}