\documentclass[../main.tex]{subfiles}
\setlength{\parskip}{0em}

\begin{document}

$H(p_1, p_2, .., p_k) < H(q_1, q_2, .., q_k)$

Исходя из $q_i = p_i, i=\overline{(3,l)}$,  неравенство принимает вид: $H(p_1, p_2, .., p_k) < H(q_1, q_2, p_3 .., p_k)$, применим свойство энтропии, чтобы свести задачу к менее емкой. 

$H(p_3,.., p_k, p_1 + p_2) + (p_1+p_2)*H(\frac{p_1}{p_1+p_2}, \frac{p_2}{p_1+p_2}) < H(p_3,.., p_k, q_1 + q_2) + (q_1+q_2)*H(\frac{q_1}{q_1+q_2}, \frac{q_2}{q_1+q_2})$

Однако нам известно, что $q_1 = p_1 + \varepsilon$, и $q_2 = p_2 - \varepsilon$, но тогда $q_1+q_2=p_1+p_2$ => \newline
$=> (p_1+p_2)*H(\frac{p_1}{p_1+p_2},\frac{p_1}{p_1+p_2}) < (p_1+p_2)*H(\frac{q_1}{p_1+p_2}, \frac{q_2}{p_1+p_2})$ =>
$H(\frac{p_1}{p_1+p_2}, \frac{p_2}{p_1+p_2}) < H(\frac{q_1}{p_1+p_2}, \frac{q_2}{p_1+p_2})$

$H(p_1, p_2)<H(p_1+\varepsilon, p_2-\varepsilon)$ (от $\frac{1}{p_1+p_2}$ можно раскрытием энтропий и взаимным уничтожением соответствующих чисел, мне лень все это перепечатывать)

Есть два пути док-ва, но первое может с некторой вероятностью не принять, из-за его недостаточной строгости.

\textbf{1)} Рассмотрим $H(p_1, p_2)<H(p_1+\varepsilon, p_2-\varepsilon)$. Так как $p_1<p_2$, то в функции $H(p_1+\varepsilon, p_2-\varepsilon), \newline
\varepsilon$ - выполняет роль <<уравнителя>>, то есть он приближает аргументы функции энтропии друг к другу $\Rightarrow$ значение функции энтропии возрастает $\Rightarrow$ наше неравнство верно.

\textbf{2)} Проанализируем неравенство $\varepsilon \leq \frac{p_2-p_1}{2}$. Если мы берем величины $\varepsilon$ меньшие чем $\frac{p_2-p_1}{2}$, исходя из $p_1 < p_2$, то при $\lim\limits_{\varepsilon \rightarrow 0}$ аргументы энтропии будут сходитьтся к $p_1, p_2$. Иначе говоря, нам достаточно рассмотреть $\lim\limits_{\varepsilon \rightarrow 0}$ в неравенстве, чтобы его доказать. 

Теперь рассмотрим наше исходное неравенство в пределах, и раскроем энтропии.

$$p_1log(p_1) + p_2log(p_2) > \lim\limits_{\varepsilon\rightarrow 0}(p_1log(p_1+\varepsilon) + p_2log(p_1-\varepsilon) + \varepsilon log(p_1+\varepsilon) - \varepsilon log(p_2-\varepsilon))$$
$$\lim\limits_{\varepsilon\rightarrow 0}(\varepsilon*(log(p_1+\varepsilon) -log(p_2-\varepsilon)))<0$$
$$(log(p_1)-log(p_2))\lim\limits_{\varepsilon\rightarrow 0}\varepsilon<0$$

Т.к.\ $p_1<p_2$, то $log(p_1)-log(p_2) <0$, а значит неравенство выполняется.

\end{document}