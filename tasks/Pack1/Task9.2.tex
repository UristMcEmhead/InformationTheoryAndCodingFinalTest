\documentclass[../main.tex]{subfiles}
\setlength{\parskip}{0em}

\begin{document}
$P(1/2, 1/2)$

В общем случае, условию будут удовлетворять любые вектора с равными компонентами, иначе говоря вектора принимающие вид $(p_1, p_2,.., p_n) = (1/n, 1/n .. , 1/n)$, по сути у нас будут равны не просто стоимости, а даже сами деревья построения кодов, т.к. мы сможем построить попарную биекцию листьев для разных кодов.

Доказательство легкое: Фано очевидно равно Хаффмена, осталось доказать равенство Шеннона любому из построенных кодов. Но если  рассмотреть таблицу построения Шеннона по кумулятивам, то получим, что этот алгоритм сходится к алгоритму Фано, а значит стоимости по разным кода равны
\end{document}