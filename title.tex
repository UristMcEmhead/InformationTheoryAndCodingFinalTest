\documentclass[../main.tex]{subfiles}

\maketitle

\textbf{Предисловие}

Если решения нет, то либо решение очевидно, либо данная задача уже была решена раньше.

В документе могут присутствовать раздличные ошибки. Кроме того, в приведенных задачах описана только суть решения, то есть некторые очевидные вещи опускались, но они могут понадобиться при устном ответе преподавателю. Также стоит помнить о том, что если вы не сможете теоретически обосновать решение, то вы получите за задачу 0 баллов.Удачи на зачете.
% Огромное спасибо Минюкову Рамилю, Хаялеевой Изиде и возможно пидору Даниле, если он сядет, за помощь в создании документа

\textbf{Ресурсы}

\noindent\href{https://drive.google.com/drive/folders/1zH1nv3Dfszxchgw9LgHNzJB2Np7RVAPX?usp=sharing}{Хранилище с билетами} \newline
\href{https://drive.google.com/drive/folders/16_fjvZ9jsglwrLPdChxKPZZxb1tk9UJ1?usp=sharing}{Материалы}\newline
\href{https://github.com/HaroldFromRovia/InformationTheoryAndCodingFinalTest}{GitHub} \newline
\href{https://www.overleaf.com/read/tzxzhqnyvmxc}{Для предложений с изменениями дока}\newline (писать комментами, либо можно написать в чат, мб увижу, однако вы всегда можете скопипастить проект себе с github)

\newpage